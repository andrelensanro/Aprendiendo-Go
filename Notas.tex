\documentclass[]{book}
\usepackage[spanish]{babel} 
\usepackage[dvipsnames]{xcolor} % Para usar colores mas variados
\usepackage{tcolorbox} % Para usar los cuadros
\usepackage{listings} % Para colocar trozos de código
\usepackage{hyperref} % Para usar href
\lstdefinestyle{customc}{ % style del trozo de código
  belowcaptionskip=1\baselineskip,
  breaklines=true,
  frame=L,
  xleftmargin=\parindent,
  language=C,
  showstringspaces=false,
  basicstyle=\footnotesize\ttfamily,
  keywordstyle=\bfseries\color{green!40!black},
  commentstyle=\itshape\color{purple!40!black},
  identifierstyle=\color{blue},
  stringstyle=\color{orange},
}
\lstdefinestyle{customhtml}{ % style del trozo de código
  belowcaptionskip=1\baselineskip,
  breaklines=true,
  frame=L,
  xleftmargin=\parindent,
  language=HTML,
  showstringspaces=false,
  basicstyle=\footnotesize\ttfamily,
  keywordstyle=\bfseries\color{LimeGreen!40!periwinkle},
  commentstyle=\itshape\color{purple!40!000cc},
  identifierstyle=\color{blue},
  stringstyle=\color{orange},
}
\hypersetup{
    colorlinks=true,
    linkcolor=blue,
    filecolor=magenta,
}
\title{\bf Aprendiendo Go}          
\author{Andrea Robles}                        
\date{enero, 2023}                          
\begin{document}
\frontmatter
\maketitle
\tableofcontents
\mainmatter
\chapter{Sintaxis de Go}
\begin{center}
\textit{En esta parte se describe el resumen o síntesis
del capítulo.}
\end{center}

\section{Asincronía y promesas en Go}
    \par La asincronía en go es distinta a diferencia de otros lenguajes. El papel de la asincronía es indicar que instruccion deberá ejecutar de manera paralela al resto de las instrucciones.

    \par Según GPT3 go maneja de la siguiente manera la asincronía: Go maneja la asincronía a través de la utilización de Goroutines y Channels. Una Goroutine es una rutina ligera que se ejecuta en paralelo con otras Goroutines en un mismo proceso. Los Canales son estructuras de datos que permiten a las Goroutines comunicarse entre sí de manera segura y sincronizada. La combinación de Goroutines y Canales permite a los programadores escribir código asíncrono de manera fácil y clara.
    \begin{tcolorbox}[
        title = Ejemplo de la asincronia en Go] 
        \begin{lstlisting}[language=go]
            import (
                "fmt"
                "strings"
                "time"
            )
            func main(){
                go miNombreLento("Andrea Selene")
            }

            func miNombreLento(nombre string){
                letras := strings.Split(nombre, "")
                for _, letra := range letras{
                    time.Sleep(time.Second)
                    fmt.Println(letra)
                }
            }
        \end{lstlisting}
    \end{tcolorbox}
\section{Canales en Go}
   \par Un canal en Go sirve para tener control de las instrucciones asincronas.
   \begin{tcolorbox}[
    title = Ejemplo de Canales en Go]
    \lstinputlisting[style=customc]{CanalesGo/main.go}

   \end{tcolorbox}
\section{Servidores}
\begin{center}
    \hyperlink{https://gowebexamples.com/http-server/}{ Ejemplos de Go en WEB.}
\end{center}
\par ¿Qué pasa con Go en la web?
\begin{itemize}
    \item Funciona como servidor o del lado del backend de endpoints
    \item Hay una función por cada endpoint que se quiera servir
    \item Se pueden manejar templates de html en go
\end{itemize}
\begin{tcolorbox}[
    title = Servidor en Go]
    \lstinputlisting[style=customc]{ServidorGo/main.go}
    
\end{tcolorbox}
% \subitem \centering{\em{\textcolor{red}{index.html}}}
\par Donde los archivos .html son los siguientes:

\begin{tcolorbox}[
    title = index.html]
    \lstinputlisting[style=customhtml]{ServidorGo/index.html}
\end{tcolorbox}

\begin{tcolorbox}[
    title = login.html
    ]
    \lstinputlisting[style=customhtml]{ServidorGo/login.html}
\end{tcolorbox}

\section{Middlewares}
\par Son interceptores que permiten ejecutar instrucciones comunes a varias funciones que reciben y devuleven los mismos tipos de variables. Esto es, se escribe un middleware comun a varias funciones. 






\end{document}


